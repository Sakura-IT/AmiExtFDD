\documentclass[10pt,a4paper]{article}
\usepackage{graphicx}
\usepackage{hyperref}
\usepackage[utf8x]{inputenc}

\begin{document}

\begin{titlepage}
\begin{center}

\includegraphics[scale=0.25]{../../common/sakuralogo.png}
\vfill

\huge
External Floppy Drive

User's Manual

\vspace*{1cm}

\normalsize
Disk drive for all Amiga systems.

\vspace*{5cm}

\today

\end{center}
\end{titlepage}

\section*{Overview}

Thank you for purchasing Sakura External Floppy Drive! This product has the following features:

\begin{itemize}
	\item Fully compatible with 880kB DD Amiga floppy disks.
	\item Reads and writes NDOS disks.
	\item Compatible with all Amiga systems featuring external floppy drive connector.
	\item Fetuares new, guaranteed floppy mechanism.
	\item Also available without the floppy drive mechanism (for installation of floppy emulators).
	\item Open source design.
	\item RoHS compliant, low power, environmentally friendly.
	\item Made by Amigans for Amigans! 
\end{itemize}

\section*{Installation}

The installation process is very easy. To install the drive perform the following steps:

\begin{itemize}
	\item Power down your Amiga.
	\item Connect the drive to external disk drive port.
	\item Power up your Amiga and enjoy additional floppy disk drive.
\end{itemize}

Your Amiga should start as normal. Upon insertion of a disk, it should automatically appear on the Workbench screen.

If your Amiga is equipped with Kickstart 2.0 or newer, it is also possible to boot from the external floppy drive (by selecting the external drive from early startup menu). However, not all track-loading demos and games support this.

Please note that this drive can not work with HD disks. It is possible to use them in DD mode if HD hole on the disk is covered. 

\section*{Technical details}

% \begin{center}
% \includegraphics{board21layout.pdf}
% \end{center}

Sakura External Floppy Drive is built around a typical PC floppy drive mechanism. It is adapted to use with the Amgia by using AmiExtFDD interface, designed by Roman Breński. The interface is extremely versatile and flexible, allowing connection of PC drives, Amiga drives and various floppy drive emulators.

\section*{Troubleshooting}

\begin{itemize}
	\item Q: Foo.
	\item A: Bar.
\end{itemize}

\section*{Acknowledgements}

Sakura External Floppy Drive was designed by Roman ,,RomanWorkshop'' Breński, Jarosław ,,jarob'' Bieliński and Radosław ,,strim'' Kujawa. 
All schematics and board layout files are licensed under Creative Commons Attribution-ShareAlike 4.0 license. 

The interface and case is made in Poland and conforms to RoHS standard. 

All new drives sold through our exclusive dealer, RetroAmi, are covered by 24 months warranty. Due to open source nature of the project, note that {\bf only} drives produced by us (and therefore sold through RetroAmi) are covered by this warranty. In case of necessary service repairs please contact the shop directly. Do not attempt to repair the drive yourself, it will void the warranty. Do not remove the sticker on the back of the drive - it contains serial number. Please save the invoice/bill as a proof of transaction.

Thanks to everyone who preordered the drive - you made this project happen!

\section*{Contact}

In case of any questions/inquires please contact RetroAmi:

\url{http://retroami.com.pl/} 

\end{document}

